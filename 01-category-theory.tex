\chapter{Basics of Category Theory}

\section{Basics}
A category $\Cscr$ consists of the following data:
\begin{enumerate}
  \item A collection of objects, denoted $\ob\Cscr$.
  \item For every pair of objects $X,Y$, a collection of arrows $\Hom(X,Y)$.
  \item A composition rule
        \[ (f,g) \mapsto f \circ g : \Hom(Y,Z) \times \Hom(X,Y) \to \Hom(X,Z). \]
  \item An identity moprhism $\one_{X} \in \Hom(X,X)$ for every object $X$.
\end{enumerate}
This data must satisfy the following axioms:
\begin{enumerate}
  \item Composition of morphisms is associative: $f \circ (g \circ h) = (f \circ g) \circ h$.
  \item The identity morphism $\one_{X}$ is a two-sided identity for composition of morphisms.
\end{enumerate}
We will usually abuse the notation and write $X \in \Cscr$ to mean that $X$ is an object of $\Cscr$, i.e.~$X$ is a member of $\ob\Cscr$.
If we wish to specify the category in $\Hom(X,Y)$, we will write $\Hom_{\Cscr}(X,Y)$ or $\Hom^{\Cscr}(X,Y)$, although $\Cscr$ will be left implicit whenever possible.
Given $X,Y \in \Cscr$, we may also write $f : X \to Y$ to mean that $f$ is a member of $\Hom(X,Y)$ when $\Cscr$ is understood from context.
Finally, we will often omit the $X$ from $\one_{X}$ and write simply $\one$ when it can be deduced from context.

We say that $\Cscr$ is \emph{locally small} if $\Hom(X,Y)$ is a set for all $X,Y \in \Cscr$.
We say that $\Cscr$ is \emph{small} if it is locally small and $\ob \Cscr$ is a set.

A morphism $f : X \to Y$ is called an \emph{isomorphism} if it has a two-sided inverse.
Explicitly, this means that there exists a morphism $g : Y \to X$ such that $f \circ g = \one$ and $g \circ f = \one$.

\begin{example}
  The category $\Set$ of sets.
  Its objects are sets, and, given two sets $S,T$, a morphism $f : S \to T$ is simply a function from $S$ to $T$.
\end{example}

\begin{example}
  The category $\Top$ of topological spaces.
  Its objects are topological spaces, and, given two spaces $X$ and $Y$, a morphism $f : X \to Y$ is a continuous map from $X$ to $Y$.
\end{example}

\begin{example}
  The category $\Group$ of groups.
  Its objects are groups, and, given two groups $G$ and $H$, a morphism $f : G \to H$ is a group homomorphism from $G$ to $H$.
\end{example}

\begin{example}
  Let $k$ be a field.
  The collection of vector spaces with $k$-linear maps between them forms a category denoted $\Vect_{k}$.
  Similarly, if $R$ is a ring, the category of (left) $R$-modules with $R$-linear morphisms forms a category denoted $R\Mod$.
\end{example}

\begin{example}
  Let $G$ be a group.
  We obtain a category $\star_{G}$ with one object $\star$ such that $\Hom(\star,\star) = G$.
  The identity $\one_{\star}$ is the identity of $G$ and composition is the multiplication in $G$.
\end{example}

\begin{example}
  Let $\Pcal$ be a poset.
  We can consider $\Pcal$ as a category with objects $\Pcal$ and
  \[ \Hom(a,b) = \begin{cases}
      {\star} & a \le b \\
      \varnothing & \text{otherwise}.
    \end{cases}\]
  Let $a,b,c,d\in \Pcal$ be given. 
  Let $f:a\to b$, $g:b\to c$, $h:c\to d$ be in $Hom(a,b)$, $Hom(b,c)$ and $Hom(c,d)$, respectively.
  Notice that $h\in Hom(x,y)$ iff $x\le y$. 
  We have $f\in Hom(a,b)$ and $g\in Hom(b,c)$, then $a\le b$ and $b\le c$; as $\Pcal$ is a poset, we have $a\le c$ and therefore $g\circ f\in Hom(a,c)$. 
  Clearly, $h\circ(g\circ f)=(h\circ g)\circ f$ since both exist.
  Now, as $b\le b$ (since $\Pcal$ is a poset), we have that $\star_b:b\to b$ is in $Hom(b,b)$. Clearly, we have that $g\circ \star_b=g$ and $\star_b\circ f=f$.
  One can think of $\star$ as the inclusion function in the partial ordering $\subseteq$ of a given set.
\end{example}

\begin{example}
  Let $\Cscr$ be a category.
  Define $\Cscr^{\op}$ as the category whose objects are $\ob \Cscr$ and whose morphisms are given by
  \[ \Hom^{\Cscr^{\op}}(X,Y) = \Hom^{\Cscr}(Y,X). \]
  Let $f\in \Hom^{\Cscr^{\op}}(X,Y)$ and $g\in \Hom^{\Cscr^{\op}}(Y,Z)$. 
  Define the composition $g\circ f$ in $\Hom^{\Cscr^{\op}}(X,Z)$ as the composition $f\circ g$ in $\Hom^{\Cscr}(Z,X)$
  This is well defined and is associative since composition and associativity are guaranteed in $\Hom^{\Cscr(Z,X)$.
  Now, the identity in $\Hom^{\Cscr^{\op}}(X,X)$ is simply the identity in $\Hom^{\Cscr}(X,X)$.
\end{example}

\section{Functors}

Let $\Cscr, \Dscr$ be two categories.
A \emph{functor}, denoted $\Fscr : \Cscr \to \Dscr$ consists of the following data:
\begin{enumerate}
  \item A function $\Fscr : \ob \Cscr \to \ob \Dscr$.
  \item For every $X,Y \in \Cscr$, a function $\Fscr : \Hom(X,Y) \to \Hom(\Fscr X, \Fscr Y)$.
\end{enumerate}
and subject to the following axioms:
\begin{enumerate}
  \item For all $X \in \Cscr$, one has $\Fscr(\one_{X}) = \one_{\Fscr X}$.
  \item For all composable morphisms $f,g$ in $\Cscr$, one has $\Fscr(f \circ g) = \Fscr(f) \circ \Fscr(g)$.
\end{enumerate}

A \emph{presheaf} $\Pscr$ on $\Cscr$ with values in $\Dscr$ is defined to be a functor $\Cscr^{\op} \to \Dscr$.
Unfolding definitions, what this means is that $\Pscr$ consists of
\begin{enumerate}
  \item A function $\Pscr : \ob\Cscr \to \ob \Dscr$.
  \item For all $X,Y \in \Cscr$, a function $\Pscr : \Hom(X,Y) \to \Hom(\Pscr(Y),\Pscr(X))$.
\end{enumerate}
which satisfies
\begin{enumerate}
  \item For all $X \in \Cscr$, $\Pscr(\one_{X}) = \one_{\Pscr X}$.
  \item For composable morphisms $f,g$, one has $\Pscr(f \circ g) = \Pscr(g) \circ \Pscr(f)$.
\end{enumerate}
Some sources use the word ``contravariant functor'' instead of ``presheaf.''

\begin{exercise}
  If $\Fscr$ is a functor and $f : X \to Y$ is an isomorphism in $\Cscr$ then $\Fscr(f)$ is an isomorphism.
\end{exercise}

\begin{example}
  If $f : G \to H$ is a morphism of groups, then we can associate to $f$ a canonical functor $\star_{G} \to \star_{H}$.
  \todo{Add details.}
\end{example}

\begin{example}
  If $f : \Pcal_{1} \to \Pcal_{2}$ is a morphism of posets, then $f$ gives rise to a functor $\Pcal_{1} \to \Pcal_{2}$ when we consider $\Pcal_{i}$ as a category.
  \todo{Add details.}
\end{example}

\begin{example}
  Let $X$ be a topological space.
  For an open set $U$, consider the vector space of real-valued continuous functions on $U$, denoted $\Ccal(U,\Rbb)$.
  Consider the collection $\Uscr_{X}$ of open subsets of $X$ as a poset with respect to inclusion, hence as a category as described above.
  For an inclusion of open sets $V \subset U$, we obtain a canonical \emph{restriction map}
  \[ \rho_{U,V}: \Ccal(U,\Rbb) \to \Ccal(V,\Rbb) \]
  defined by sending a continuous function on $U$ to its restriction to $V$.
  This construction defines a \emph{presheaf} on $\Uscr_{X}$ with values in $\Vect_{\Rbb}$.
\end{example}

\begin{example}
  Let $\Cscr$ be a locally small category and let $X$ be an object of $\Cscr$.
  We define a functor $h^{X} : \Cscr \to \Set$ as follows.
  For $Y, Z \in \Cscr$, $f : Y \to Z$, and $g \in \Hom(X,Y)$, we put
  \[ h^{X}(Y) = \Hom(X,Y), \ h^{X}(f)(g) = f \circ g \in \Hom(X,Z). \]
  \exer{It is straightforward to check that $h^{X}$, defined in this way, is indeed a functor.}

  Similarly, we define a presheaf $h_{X}$ on $\Cscr$ with values in $\Set$ as follows.
  For $Y,Z \in \Cscr$, $f : Y \to Z$ and $g \in \Hom(Z,X)$, we put
  \[ h_{X}(Y) = \Hom(Y,X), \ h_{X}(f)(g) = g \circ f \in \Hom(Y,X). \]
  \exer{It is straightforward to check that $h_{X}$, defined in this way, is indeed a presheaf.}
\end{example}

\begin{example}[The category of categories]
  \todo{Add definition of composition of functors and identity functors.}
  The category of categories has objects consisting of (small) categories and functors as morphisms.
\end{example}

\section{Natural transformations}

\begin{definition}
Suppose that $\Ccal$ and $\Dcal$ are categories and let $F,G:\Ccal\to\Dcal$ be functors. 
We define a \textit{natural transformation} as a rule $\zeta:\Fscr\to\Gscr$ which assigns a morphism $\zeta_X:\Fscr(X)\to \Gscr(X)$ to every $X\in \Ccal$.
This rule must satisfy that for every morphism $f:X\to Y$ in $Hom(X,Y)$, we have that the diagram

\begin{tikzcd}
\Fscr(X) \arrow[r, "\Fscr(f)"] \arrow[d, "\zeta_X"] & \Fscr(Y) \arrow[d, "\zeta_Y"] \\
\Gscr(X) \arrow[r, "\Gscr(f)"]                      & \Gscr(Y)                     
\end{tikzcd}

commutes, that is, $\zeta_Y\circ \Fscr(f)=\Gscr(f)\circ\zeta_X$.
\end{definition}

\todo{Define functor categories.}

\section{The Yoneda Lemma}

\todo{State and prove the Yoneda lemma.}

%%% Local Variables:
%%% mode: latex
%%% TeX-master: "main"
%%% End:
