\chapter{Basics of Category Theory}

\section{Basics}
A category $\Cscr$ consists of the following data:
\begin{enumerate}
  \item A collection of objects, denoted $\ob\Cscr$.
  \item For every pair of objects $X,Y$, a collection of arrows $\Hom(X,Y)$.
  \item A composition rule
        \[ (f,g) \mapsto f \circ g : \Hom(Y,Z) \times \Hom(X,Y) \to \Hom(X,Z). \]
  \item An identity moprhism $\one_{X} \in \Hom(X,X)$ for every object $X$.
\end{enumerate}
This data must satisfy the following axioms:
\begin{enumerate}
  \item Composition of morphisms is associative: $f \circ (g \circ h) = (f \circ g) \circ h$.
  \item The identity morphism $\one_{X}$ is a two-sided identity for composition of morphisms.
\end{enumerate}
We will usually abuse the notation and write $X \in \Cscr$ to mean that $X$ is an object of $\Cscr$, i.e.~$X$ is a member of $\ob\Cscr$.
If we wish to specify the category in $\Hom(X,Y)$, we will write $\Hom_{\Cscr}(X,Y)$ or $\Hom^{\Cscr}(X,Y)$, although $\Cscr$ will be left implicit whenever possible.
Given $X,Y \in \Cscr$, we may also write $f : X \to Y$ to mean that $f$ is a member of $\Hom(X,Y)$ when $\Cscr$ is understood from context.
Finally, we will often omit the $X$ from $\one_{X}$ and write simply $\one$ when it can be deduced from context.

A morphism $f : X \to Y$ is called an \emph{isomorphism} if it has a two-sided inverse.
Explicitly, this means that there exists a morphism $g : Y \to X$ such that $f \circ g = \one$ and $g \circ f = \one$.

\begin{example}
  The category $\Set$ of sets.
  Its objects are sets, and, given two sets $S,T$, a morphism $f : S \to T$ is simply a function from $S$ to $T$.
\end{example}

\begin{example}
  The category $\Top$ of topological spaces.
  Its objects are topological spaces, and, given two spaces $X$ and $Y$, a morphism $f : X \to Y$ is a continuous map from $X$ to $Y$.
\end{example}

\begin{example}
  The category $\Group$ of groups.
  Its objects are groups, and, given two groups $G$ and $H$, a morphism $f : G \to H$ is a group homomorphism from $G$ to $H$.
\end{example}

\begin{example}
  Let $G$ be a group.
  We obtain a category $\star_{G}$ with one object $\star$ such that $\Hom(\star,\star) = G$.
  The iddentity $\one_{\star}$ is the identity of $G$ and composition is the multiplication in $G$.
\end{example}

\begin{example}
  Let $\Pcal$ be a poset.
  We can consider $\Pcal$ as a category with objects $\Pcal$ and
  \[ \Hom(a,b) = \begin{cases}
      {\star} & a \le b \\
      \varnothing & \text{otherwise}.
    \end{cases}\]
  \todo{Add details about composition and identities.}
\end{example}

\begin{example}
  Let $\Cscr$ be a category.
  Define $\Cscr^{\op}$ as the category whose objects are $\ob \Cscr$ and whose morphisms are given by
  \[ \Hom^{\Cscr^{\op}}(X,Y) = \Hom^{\Cscr}(Y,X). \]
  \todo{Add details about composition and identities.}
\end{example}

\section{Functors}

Let $\Cscr, \Dscr$ be two categories.
A \emph{functor}, denoted $\Fscr : \Cscr \to \Dscr$ consists of the following data:
\begin{enumerate}
  \item A function $\Fscr : \ob \Cscr \to \ob \Dscr$.
  \item For every $X,Y \in \Cscr$, a function $\Fscr : \Hom(X,Y) \to \Hom(\Fscr X, \Fscr Y)$.
\end{enumerate}
and subject to the following axioms:
\begin{enumerate}
  \item For all $X \in \Cscr$, one has $\Fscr(\one_{X}) = \one_{\Fscr X}$.
  \item For all composable morphisms $f,g$ in $\Cscr$, one has $\Fscr(f \cric g) = \Fscr(f) \circ \Fscr(g)$.
\end{enumerate}

A \emph{presheaf} $\Pscr$ on $\Cscr$ with values in $\Dscr$ is defined to be a functor $\Cscr^{\op} \to \Dscr$.
Unfolding definitions, what this means is that $\Pscr$ consists of
\begin{enumerate}
  \item A function $\Pscr : \ob\Cscr \to \ob \Dscr$.
  \item For all $X,Y \in \Cscr$, a function $\Pscr : \Hom(X,Y) \to \Hom(\Pscr(Y),\Pscr(X))$.
\end{enumerate}
which satisfies
\begin{enumerate}
  \item For all $X \in \Cscr$, $\Pscr(\one_{X}) = \one_{\Pscr X}$.
  \item For composable morphisms $f,g$, one has $\Pscr(f \circ g) = \Pscr(g) \circ \Pscr(f)$.
\end{enumerate}
Some sources use the word ``contravariant functor'' instead of ``presheaf.''

%%% Local Variables:
%%% mode: latex
%%% TeX-master: "main"
%%% End:
